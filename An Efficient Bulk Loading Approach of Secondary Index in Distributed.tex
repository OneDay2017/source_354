
%%%%%%%%%%%%%%%%%%%%%%% file typeinst.tex %%%%%%%%%%%%%%%%%%%%%%%%%
%
% This is the LaTeX source for the instructions to authors using
% the LaTeX document class 'llncs.cls' for contributions to
% the Lecture Notes in Computer Sciences series.
% http://www.springer.com/lncs       Springer Heidelberg 2006/05/04
%
% It may be used as a template for your own input - copy it
% to a new file with a new name and use it as the basis
% for your article.
%
% NB: the document class 'llncs' has its own and detailed documentation, see
% ftp://ftp.springer.de/data/pubftp/pub/tex/latex/llncs/latex2e/llncsdoc.pdf
%
%%%%%%%%%%%%%%%%%%%%%%%%%%%%%%%%%%%%%%%%%%%%%%%%%%%%%%%%%%%%%%%%%%%

\documentclass[runningheads,a4paper]{llncs}
\usepackage{clrscode}
\setcounter{tocdepth}{3}
\setcounter{secnumdepth}{3}
\usepackage{amsmath,amssymb}
\usepackage{graphicx}
\usepackage{pdfpages}
\usepackage{indentfirst}
%\usepackage[boxed,commentsnumbered]{algorithm2e}
\usepackage[ruled,vlined]{algorithm2e}
\usepackage{enumerate}
\usepackage{comment}
\usepackage{url}
\usepackage{enumitem}
\usepackage{tikz,pgfplots}
\usepackage{cite}
\urldef{\mailsa}\path|{yanchaozhu}@stu.ecnu.edu.cn, |
\urldef{\mailsb}\path|{zhzhang,pcai,wnqian,ayzhou}@sei.ecnu.edu.cn|
\newcommand{\keywords}[1]{\par\addvspace\baselineskip
\noindent\keywordname\enspace\ignorespaces#1}


\begin{document}
\pagestyle{empty}
\mainmatter  % start of an individual contribution

% first the title is needed
\title{A Distributed Fault-tolerant Secondary Index on
	PBFT-based Block Chain
}


\urldef{\mailsa}\path|{yczhu}@stu.ecnu.edu.cn, |
\urldef{\mailsb}\path|{zhzhang,pcai,wnqian,ayzhou}@sei.ecnu.edu.cn|


% a short form should be given in case it is too long for the running head
\titlerunning{Bulk Loading}
% the name(s) of the author(s) follow(s) next
%
% NB: Chinese authors should write their first names(s) in front of
% their surnames. This ensures that the names appear correctly in
% the running heads and the author index.
%
%
% NB: a more complex sample for affiliations and the mapping to the
% corresponding authors can be found in the file "llncs.dem"
% (search for the string "\mainmatter" where a contribution starts).
% "llncs.dem" accompanies the document class "llncs.cls".
%
\author{Yanchao Zhu \inst{1}
	\and Zhao Zhang \thanks{Corresponding author.} \inst{1,2} \and Peng Cai \inst{1} \and Weining Qian \inst{1} \and\\
	Aoying Zhou \inst{1}}
%
\authorrunning{Yc. Zhu, Z. Zhang, P. Cai, et al}
% (feature abused for this document to repeat the title also on left hand pages)

% the affiliations are given next; don't give your e-mail address
% unless you accept that it will be published
\institute{ School of Data Science and Engineering, 
	\and School of Computer Science and Software Engineering, \\ East China Normal University \\
	\mailsa\\
	\mailsb\\
}

%
% NB: a more complex sample for affiliations and the mapping to the
% corresponding authors can be found in the file "llncs.dem"
% (search for the string "\mainmatter" where a contribution starts).
% "llncs.dem" accompanies the document class "llncs.cls".
\toctitle{Lecture Notes in Computer Science}
\tocauthor{Authors' Instructions}
\maketitle
%52151500009@ecnu.cn ÃÜÂë:stayfoolish123
%https://cmt.research.microsoft.com/DASFAA2017/


\begin{abstract}



\keywords{}
\end{abstract}


\section{Introduction}
\vspace{-0.2cm}
Recently, block chain is becoming more and more popular
for de-centered application, such as Bitcoin, Ethereum and
Hyperledger. These systems tolerate the Byzantine General
Problem, which means they can guarantee the correctness
of system even when these are several bad nodes in the system. In practice, these systems are indeed required for applications such as decentralized currencies, tracking use of
charitable funds and Intelligent contract.
In these systems, data is stored in database. These systems use consensus algorithm such as PBFT, PoW and PoS
for consensus before querying or modifying the state of the
system. Applications interact with the system using intelligent contract which is stored on each node(store procedure).
The system must scan the whole data set distributed in order to answer queries on non-primary key attributes. With
the increment of size of data, query performance will become worser. In practice, building a secondary index is a
popular solution to decrease response time of the queries
on non-primary key attributes. In block chain systems, all
data are stored on each node, building indexes on a single
node will occupy lots of storage space. What’s more, single
node is not able to provide service for concurrent business
query. To the best of our knowledge, there are no approach
for distributed secondary index on block chain. The major
difficulties include (1) (2)

We have made the following contributions in this paper:
\begin{itemize}
\renewcommand{\labelitemi}{$\bullet$}
\item
 ssssssssssss
\item
  sssssssssssssssssss
\item
 sssssssssssssssssss
  %the approach is able to efficiently do bulk loading of index by taking rational use of the resources of cluster and guarantee the load balance of secondary index.
\end{itemize}

The rest of paper is organized as follows: Section 2 introduces background knowledge related to our work. Section 3 gives an overview of index construction. Discussion of details of bulk loading is in Section 4. We then provide an evaluation of the approach in Section 5. Finally, we describe related work in Section 6, and conclude our paper in Section 7.

\vspace{-0.2cm}

\section{Background}


\section{Index Structure}
\vspace{-0.2cm}
A common structure of an index is an index table, which is partitioned and stored on a cluster of nodes as an ordinary one. A record in the index table is combination of an index column \emph{(searchkey)} and a primary key column of the data table, like as  \emph{(searchkey,primarykey)}. Figure 2 shows an example of the index. The primary key of the item table is column \emph{ItemId}. If we create an index on the column \emph{Sale}, the schema of index table is shown in Figure 2. A query is executed by accessing the index table to get the primary key of the data table, and then getting results from the data table according to the aforementioned primary key. Certainly, we also permit users to build the covering index, which is an index that contains all, and possibly more, the columns that you need for your query.We use a common partition strategy of hash for index partition.



\section{Index Operation}
\subsection{Index Insertion}
\vspace{-0.2cm}
For index insertion with index value V, a node need to judge if it need to modify index table. If the node store the
index of value V, it will update the incremental cryptography of index value V for query validation. Insert operation
consists of steps shown in Algorithm 1: Line 6-judge if the node needs to store the index value. Line 7-add index entry
into index table. Line 8 to line 9 -update the incremental cryptography of index value V. Since each node is store all
data of the system. Modify of index will not cause distributed transactions which is important to a distributed system.

\SetKwProg{Fn}{\underline{Function}}{}{end}
%        \newcommand{\forcond}{$i=0$ \KwTo $n$}
\SetKwFunction{localindexconstruct}{IndexInsertion}
%\SetAlgorithmName{Figure}{}
\LinesNumbered
%           \newcommand{\fortol}{$i=0$ \KwTo $L$}
%           \newcommand{\fortok}{$j=1$ \KwTo $k$}
\vspace{-0.5cm}
\begin{algorithm}[htb]
	\SetAlgoLined
	\caption{Index Insertion}%
	Let $H$ denote a hash map of incremental crypto of index value\;
	Let $S$ denote value of key V in hash map\;
	Let $N$ denote the number of index partition\;
	Let $ID$ denote the number of node\;
	%\Fn(\tcc*[h]{handle request for index records}){\indexrequest{range}}
	\Fn{\localindexconstruct{index\_value,rowkey}}{	
		\tcc*[h]{local index construction}
		
		\If{$ID==hash_value(value)$}
		{
			$add_index(value,rowkey)$\;
			$S=H.get(value)$\;
			$S.add_incremental(rowkey)$\;
		}
	}
\end{algorithm}


\subsection{Index Deletion \& Update}
\vspace{-0.2cm}
For index deletion with index value V, a node need to judge if it need to modify the index table. If the node store
the index of value V, it will update the incremental cryptography of index value V for query validation. Insert operation
consists of steps shown in Algorithm 1: Line 2-judge if the node needs to store the index value. Line 3 to Line 4-delete
index entry from index table. Since each node is store all data of the system. Modify of index will not cause distributed transactions which is important to a distributed system.

\SetKwProg{Fn}{\underline{Function}}{}{end}
%        \newcommand{\forcond}{$i=0$ \KwTo $n$}
\SetKwFunction{localindexconstruct}{IndexDeletion}
%\SetAlgorithmName{Figure}{}
\LinesNumbered
%           \newcommand{\fortol}{$i=0$ \KwTo $L$}
%           \newcommand{\fortok}{$j=1$ \KwTo $k$}
\vspace{-0.5cm}
\begin{algorithm}[htb]
	\SetAlgoLined
	\caption{Index Deletion}%

	%\Fn(\tcc*[h]{handle request for index records}){\indexrequest{range}}
	\Fn{\localindexconstruct{index\_value,rowkey}}{	
		\tcc*[h]{delete index}
		
		\If{$ID==hash_value(value)$}
		{
			$delete_index(value,rowkey)$\;
			$S=H.get(value)$\;
			$delete_incremental(S,rowkey)$\;
		}
	}
\end{algorithm}


\section{Query}
\vspace{-0.2cm}
In order to get correct data, the client need to send a
query request to the block chain, which means a query request need N 2 times of network communication. By adding the trusty node, a query just need 2 times of network communication. The query operation of the index consists of three part: query route, index query and query validation.

\SetKwProg{Fn}{\underline{Function}}{}{end}
%        \newcommand{\forcond}{$i=0$ \KwTo $n$}
\SetKwFunction{localindexconstruct}{IndexUpdate}
%\SetAlgorithmName{Figure}{}
\LinesNumbered
%           \newcommand{\fortol}{$i=0$ \KwTo $L$}
%           \newcommand{\fortok}{$j=1$ \KwTo $k$}
\vspace{-0.5cm}
\begin{algorithm}[htb]
	\SetAlgoLined
	\caption{Index Update}%
	
	%\Fn(\tcc*[h]{handle request for index records}){\indexrequest{range}}
	\Fn{\localindexconstruct{old\_value,new\_value,rowkey}}{	
		\tcc*[h]{update index}
		
		\If{$ID==hash_value(old_value)$}
		{
			$delete_index(old_value,rowkey)$\;
			$S=H.get(old_index_value)$\;
			$S.delete_incremental(old_index_value)$\;
		}
	
	    \If{$ID==hash_value(new_index_value)$}
	    {
	    	$add_index(new_value,rowkey)$\;
	    	$S=H.get(new_index_value)$\;
	    	$S.add_incremental(rowkey)$;
    	}
		
	}
\end{algorithm}



Algorithm 4 describe the process of query route, the trusty node will send the request to the node storing the index entry. The node storing the index entry will search local index and return the result along with the incremental cryptography, the trusty node will check the cryptography for validation.



When a node receive a request for query of index, it will search local index for query. Since index is distributed on
several node, each node maintain a partition of index, the query process is efficient. Line 2 to line 6 in Algorithm5
describe the process of query on a node.

\SetKwProg{Fn}{\underline{Function}}{}{end}
%        \newcommand{\forcond}{$i=0$ \KwTo $n$}
\SetKwFunction{localindexconstruct}{IndexQuery}
%\SetAlgorithmName{Figure}{}
\LinesNumbered
%           \newcommand{\fortol}{$i=0$ \KwTo $L$}
%           \newcommand{\fortok}{$j=1$ \KwTo $k$}
\vspace{-0.5cm}
\begin{algorithm}[htb]
	\SetAlgoLined
	\caption{Index Query}%
	
	%\Fn(\tcc*[h]{handle request for index records}){\indexrequest{range}}
	\Fn{\localindexconstruct{value}}{	
		\tcc*[h]{query by index}
		
		\If{$ID==hash_value(value)$}
		{
			$S=H.get(value)$\;
			$V=get_rowkey(value)$\;
		}
	    $send_result(value,V)$\;
	}
\end{algorithm}


Since a node storing an index partition may be faulty, the trusty node will check the response from the node. If the cryptography of result is not the same as the result on the trusty node, the trusty node will search local index for
response. The validation process handled in the following way: the trusty node will first calculate the cryptography of
the return result of an index value, then it will compare the cryptography with the cryptography of its own for the value.
If the two cryptographies are same, the result is correct, else, the result is wrong. Line 2 to line 3 of Algorithm6 describe
the process of validation and line 7 to line 10 describe the process of searching local index of the trusty node when the
return result is wrong.
 
As we can see, the trusty node just need to check the result and search local index when the result is wrong.


\SetKwProg{Fn}{\underline{Function}}{}{end}
%        \newcommand{\forcond}{$i=0$ \KwTo $n$}
\SetKwFunction{localindexconstruct}{QueryValidation}
%\SetAlgorithmName{Figure}{}
\LinesNumbered
%           \newcommand{\fortol}{$i=0$ \KwTo $L$}
%           \newcommand{\fortok}{$j=1$ \KwTo $k$}
\vspace{-0.5cm}
\begin{algorithm}[htb]
	\SetAlgoLined
	\caption{Query Validation}%
	
	%\Fn(\tcc*[h]{handle request for index records}){\indexrequest{range}}
	\Fn{\localindexconstruct{value,V}}{	
		\tcc*[h]{validate result of query result}
		
		$S=H.get(value)$\;
		$correct=validate(S,V)$\;
		\If{correct}
		{
			$return_result V$\;
		}
	    \Else
	    {
	    	$V=search_local_index(value)$\;
	    	$return_result V$\;
	    }
	}
\end{algorithm}
\section{Trusty Node}
\vspace{-0.2cm}
In a PBFT-based system, at most n/3 nodes are faulty. Since index is distributed on several nodes, nodes storing
index partitions may be faulty, it may not return result, or return error message. In order to get correct result by index,
a naive approach is first to construct index on each node, and then handle a query with the PBFT protocol. However the
approach need to store index on all nodes and each query need n2 times of network communication. In our approach, we use a trusty node for data validation. For each query, the trusty node checks the result from index partition. If the result is correct after validating, the trusty node will return result,else it will search its own local index for response. By adding a trusty node, we avoid acquiring result by PBFT protocol which takes much network resource, which means our approach is more efficient.

Since the trusty node need to check the result returned by other nodes, it must be contain all index value. For each insertion of index, the PBFT cluster can apply change to the state machine if and on if the change has been made to the
trusty node. The property of insertion guarantee the trusty node to contain all the changes to the state machine. In
order check the result of query quickly, we use the technology of incremental cryptography. For each index entry, we
maintain a cryptography for it, for modifications of index, we modify the cryptography for that index entry. For validating the result of query by index, the trusty just need to calculate the cryptography of returned result instead searching local index. If the cryptography of returned result is the same as the cryptography on trusty node, the result can be
returned to client, else the trusty node need to search local index for response.



\section{Experimental Evaluation}

\vspace{-0.2cm}
\section{Related Work}


\section{Conclusion}

\vspace{-0.3cm}
\bibliographystyle{abbrv}
\bibliography{indexref}

\end{document} 