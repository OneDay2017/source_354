\section{Introduction}
\vspace{-0.2cm}
Recently, block chain is becoming more and more popular
for de-centered application, such as Bitcoin, Ethereum and
Hyperledger. These systems tolerate the Byzantine General
Problem, which means they can guarantee the correctness
of system even when these are several bad nodes in the system. In practice, these systems are indeed required for applications such as decentralized currencies, tracking use of
charitable funds and Intelligent contract.
In these systems, data is stored in database. These systems use consensus algorithm such as PBFT, PoW and PoS
for consensus before querying or modifying the state of the
system. Applications interact with the system using intelligent contract which is stored on each node(store procedure).
The system must scan the whole data set distributed in order to answer queries on non-primary key attributes. With
the increment of size of data, query performance will become worser. In practice, building a secondary index is a
popular solution to decrease response time of the queries
on non-primary key attributes. In block chain systems, all
data are stored on each node, building indexes on a single
node will occupy lots of storage space. What’s more, single
node is not able to provide service for concurrent business
query. To the best of our knowledge, there are no approach
for distributed secondary index on block chain. The major
difficulties include (1) (2)

We have made the following contributions in this paper:
\begin{itemize}
\renewcommand{\labelitemi}{$\bullet$}
\item
 ssssssssssss
\item
  sssssssssssssssssss
\item
 sssssssssssssssssss
  %the approach is able to efficiently do bulk loading of index by taking rational use of the resources of cluster and guarantee the load balance of secondary index.
\end{itemize}

The rest of paper is organized as follows: Section 2 introduces background knowledge related to our work. Section 3 gives an overview of index construction. Discussion of details of bulk loading is in Section 4. We then provide an evaluation of the approach in Section 5. Finally, we describe related work in Section 6, and conclude our paper in Section 7.

\vspace{-0.2cm}
