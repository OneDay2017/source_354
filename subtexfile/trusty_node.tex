\section{Trusty Node}
\vspace{-0.2cm}
In a PBFT-based system, at most n/3 nodes are faulty. Since index is distributed on several nodes, nodes storing
index partitions may be faulty, it may not return result, or return error message. In order to get correct result by index,
a naive approach is first to construct index on each node, and then handle a query with the PBFT protocol. However the
approach need to store index on all nodes and each query need n2 times of network communication. In our approach, we use a trusty node for data validation. For each query, the trusty node checks the result from index partition. If the result is correct after validating, the trusty node will return result,else it will search its own local index for response. By adding a trusty node, we avoid acquiring result by PBFT protocol which takes much network resource, which means our approach is more efficient.

Since the trusty node need to check the result returned by other nodes, it must be contain all index value. For each insertion of index, the PBFT cluster can apply change to the state machine if and on if the change has been made to the
trusty node. The property of insertion guarantee the trusty node to contain all the changes to the state machine. In
order check the result of query quickly, we use the technology of incremental cryptography. For each index entry, we
maintain a cryptography for it, for modifications of index, we modify the cryptography for that index entry. For validating the result of query by index, the trusty just need to calculate the cryptography of returned result instead searching local index. If the cryptography of returned result is the same as the cryptography on trusty node, the result can be
returned to client, else the trusty node need to search local index for response.


\subsection{Combination of PBFT \& RAFT}
\vspace{-0.2cm}

As is mentioned before, in order to check the returned result of query, the PBFT cluster can not apply changes to state machine before receiving acknowledgement from the trusty node. However, if the trusty node crashes, the system will block until recovery of the trusty node which is intolerant. Thus, the system need to provide service even when the trusty node crashes. Consider the situation in a blockchain, some nodes are high-level and some nodes are common, common nodes may send wrong messages for profit while high-level nodes are trusty and do not send wrong messages. Consider the property, we divide nodes in a blockchain into two partitions, on partition use the consensus protocol of PBFT and other nodes use the protocol of RAFT. The division of nodes in a blockchain on the one hand can improve the performance of blcokchain, on the other hand, since high-level nodes are trusty and can be organized as a RAFT cluster, it can play the role of trusty node in section 6.

In order to guarantee the consistency of the state machine on each node, we need to combine nodes using protocol of RAFT and nodes using protocol of PBFT. In the following of this section, we will show the protocol combine PBFT and RAFT.

\subsubsection{The Algorithm}

The key point of combining PBFT with RAFT is to apply changes to PBFT and RAFT in an atomic way. In order to achieve the goal, we extend PBFT to commit if and only if RAFT has applied changes. We extend RAFT leader to send log according to the sequence number in PBFT, the leader og RAFT does not send out an log until it has received enough commit info from PBFT.

The algorithm works roughly as follows:
